
\chapter{Compatibility}
\label{compat}
\section{General remarks}

X11-Basic deviates in numerous aspects from ANSI BASIC. It in event is also
different from GfA-Basic (Atari ST) all though it tries to be compatible and
really looks similar:

\subsection*{ELSE IF vs. ELSEIF}

This interpreter uses the ELSE IF form of the "else if" statement with a space 
between ELSE and IF. In contrast, ANSI BASIC uses ELSEIF and END IF. Other
interpreters may also use the combination ELSEIF and END IF.


\subsection*{Local variables}

Local variables must be declared local in the procedure/function. Any other
variables are treated as global.

\subsection*{Call By-Value vs. By-Reference}

Variables in a GOSUB statement as in \verb|GOSUB test(a)| are passed "by-value" to the
PROCEDURE: the subroutine gets the value but can not change the variable from which the value
came from. To pass the variable "by-reference", use the VAR keyword as in "GOSUB test(VAR 
a)": the subroutine then not only gets the value but the variable itself and can change it
(for more information, see the documentation of the GOSUB  statement). The same rules apply
to FUNCTION: VAR in the parameter list of a function call allows a FUNCTION to get a variable
parameter "by-reference". In contrast, traditional BASIC interpreters always pass variables
in  parameter lists "by-reference". The problem with "by-reference" parameters is that you
must be fully aware of what happens inside the subroutine: assignments to parameter variables
inside the subroutine might change the values of variables in the calling line.


\subsection*{Assignment operator}

X11-BASIC does not have an assignment operator but overloads the equal sign to
act as the assignment operator or as comparison operator depending on context:
In a regular expression, all equal signs are considered to be the comparison
operator, as in \verb|IF (a=2)|. However, in an "assignment-style" expression
(as in \verb|LET a=1|), the first equal sign is considered to be the assignment
operator. Here is an example which assigns the result of a comparison (TRUE or
FALSE) to the variable <a> and thus shows both forms of usage of the equal sign:

\begin{verbatim}
a=(b=c)
\end{verbatim}


\subsection*{Assignments to modifiable l(eft)value}

Some implementations of BASIC allow the use of functions on the left side of
assignments as in \verb|MID$(a$,5,1)="a"|. X11-Basic does not support this
syntax but requires a variable (a "modifiable lvalue") on the left side of such
expressions.

\subsection*{INT() function}

In X11-Basic INT() gives probably different results for negative numbers and for numbers
bigger than  2147483648. INT() is internally implemented as "cast to int" (has ever be like
this, very fast, and also the compiler relies on this). This means, that the argument must
not contain numbers which cannot be converted to 32bit integers. And also the fractional part
of the floating point numbers is cut off (like TRUNC()) instead of rounded down (like on most
other BASIC dialects). If you rely on a correct behaviour for negative numbers and for big
numbers you probably want to use FLOOR() instead.

\subsection*{DIM Statement}

In X11-Basic the DIM statement probably behaves different compared to other  dialects of
BASIC. DIM in X11-Basic will reserve space in memory for exacly the number of indexes
specified.  Other BASIC dialect do reserve one more than specified. If you are surprised
getting "Field index out of range" errors, this probably comes from accessing a field index
which is not there. For example: DIM a(5) will reserve memory for exactly 5 values: a(0),
a(1), a(2), a(3), and a(4). a(5) does not exist and therefor you will get an error if you try
to access it. The way X11-Basic implemented it is more logical and similar to C and JAVA. But
if you are used to thinking that an array starts with the first index 1 (instead of 0) you
will probably be a little confused.

\subsection*{LET Statement}

Al though it is implemented into X11-Basic, there is no benefit in  using the
LET statement. On contrary, using LET makes your program  slower than necessary.
Just leave it out, do not use it. Assignments can be made without the LET
statement.


\subsection*{TOS/GEM implementation}

Because Gfa-Basic on ATARI-ST makes much use of the built in GUI functions of
the ATARI ST, which are not available on other operating systems, X11-Basic can
only have limited compatibility. GEM style (and compatible) ALERT boxes, menus 
and object trees are supported by X11-Basic and can be used in a similar way.
Even ATARI ST *.rsc files can be loaded. But other functions like LINEA
functions, the VDISYS, GEMSYS, BIOS, XBIOS and GEMDOS calls are not possible.
Also many other commands are not implemented because the author thinks that they
have no useful effect on UNIX platforms. Some might be included in a later
version of X11-Basic (see the list below). Since many GfA-Basic programs make
use of more or less of these functions, they will have to be modified before
they can be run with X11-Basic.

\subsection*{The INLINE statement}

The INLINE statement is not supported, because the  source code of X11-Basic 
programs is pure ASCII text. But an alternative has been implemented. (see
\verb|INLINE$()|).

\subsection*{Incompatible data types}

X11-Basic uses the default datatype (without suffix) and the integer data type (
Suffix \%). This is compatible with most of the BASIC dialects. However the
complex data type (suffix \#) is not supported by most of the BASIC dialects and the 
suffix \# is sometimes optionally used by regular float variables 
(like in GFA-Basic).

The suffix \& which is used for big integer variables could be confused with the 
short int data type of X11-Basic which also uses this suffix. However, in 
general these programs will run and give correct results. Using the 
infinite precision routines is just slower.

Short int and byte data types will not be used by X11-Basic. There used be 
useful only on computers with short memory do save some RAM. 
These times have passed, so that the statndard integer data type 
(with the suffix \%) will do.

The suffix $|$ will be reserved for future use, most likely for multiple
precision floating point variables.


\section{GFA-Basic compatibility}
\label{gfacompat}

Following GFA-Basic commands and functions are not supported and probably never
will be. Most of them are obsolete on UNIX systems. When porting from GFA-Basic
to X11-Basic, they have to be removed or replaced by an alternative routine:

\paragraph{obsolete, because there is an alternative function in X11-Basic:}

\begin{tabbing}
XXXXXXXXXXXXXX\=XXXXXXXXXXXXXXXXXX\=\kill\\
\verb|==|\>		Comparison operator for approximately equal  --> =\\
\verb|ARECT, ATEXT|\>  LINE-A functions --> \verb|BOX, TEXT|\\
\verb|ALINE, HLINE|\>  LINE-A functions --> \verb|LINE|\\
\verb|ACHAR, ACLIP|\> --> \verb|TEXT, CLIP|\\
\verb|APOLY|       \> --> \verb|POLY|\\
\verb|COSQ(), SINQ()|\>	quick table based cosine/sine --> \verb|COS(), SIN()|\\
\verb|DIR|\>		Lists the files on a disc.  --> \verb|SYSTEM "ls"|\\
\verb|DRAW|\>	      Draws points and lines. --> \verb|PLOT, LINE|\\
\verb|FILES|\>		Lists the files on a disk.  --> \verb|SYSTEM "ls -l"|\\
\verb|FRE()|\>	        Returns the amount of memory free (in bytes). --> see below\\
\verb|MAT CLR|\>        clears a matrix/makes a zero matrix --> \verb|ARRAYFILL, CLR, 0()|\\
\verb|MAT DET|\>        calculates the determinat of a matrix --> \verb|DET()|\\
\verb|MAT INV|\>        calculates the inverse of a matrix --> \verb|INV()|\\
\verb|MAT ONE|\>        creates a unit matrix --> \verb|1()|\\
\verb|MAT TRANS|\>      calculates the transverse of a matrix --> \verb|TRANS()|\\
\verb|MSHRINK()|\>      Reduces the size of a storage area --> \verb|REALLOC()|\\
\verb|NAME AS|\>        Renames an existing file. --> \verb|RENAME|\\
\verb|QSORT|\>  	      Sorts  the elements of an array.  --> \verb|SORT|\\
\verb|RC_COPY|\>   	Copies rectangular screen sections  (--> \verb|COPYAREA|)\\
\verb|RESERVE|\>      	Increases or decreases the memory used by basic (obsolete)\\
\verb|RND as a sysvar|\>  see \verb|RND()|\\
\verb|ROL&(), ROL%()|\>  Rotates a bit pattern left. --> \verb|ROL()|\\
\verb|ROR&(), ROR%()|\>  Rotates a bit pattern right. --> \verb|ROR()|\\
\verb|SHEL_FIND()|\>    $\longrightarrow$ \verb|SYSTEM "find ..."|\\
\verb|SHL&(), SHL%()|\>  Shifts a bit pattern left --> \verb|SHL()|\\
\verb|SHR&(), SHR%()|\>  Shifts a bit pattern left --> \verb|SHR()|\\
\verb|SYSTEM |\> 	obsolete  $\longrightarrow$ \verb|QUIT|\\
\verb|SHEL_ENVRN()|\>	$\longrightarrow$ \verb|ENV$()|\\
\verb|SHEL_READ |\> 	obsolete  $\longrightarrow$ \verb|PARAM$()|\\
\verb|SSORT|\>        Sorts using the Shell-Metzner method. --> \verb|SORT|\\
\verb|THEN|\>           keyword in If statements (obsolete)\\
\end{tabbing}

For some GFA-Basic commands you can construct replacement functions in 
X11-Basic like:
\begin{mdframed}[hidealllines=true,backgroundcolor=blue!20]
\begin{verbatim}
' Get the free memory available (in Bytes)
' n=0 physical memory
' n=1 Swap space
FUNCTION fre(n)
  LOCAL a,t$,a$,unit$,s$
  IF n=0
    s$="MemFree:"
  ELSE 
    s$="SwapFree:"
  ENDIF
  a=FREEFILE()
  OPEN "I",#a,"/proc/meminfo"
  WHILE NOT EOF(#a)
    LINEINPUT #a,t$
    EXIT IF word$(t$,1)=s$
  WEND
  CLOSE #a
  t$=TRIM$(t$)
  a$=word$(t$,2)
  unit$=word$(t$,3)
  IF unit$="kB"
    RETURN VAL(a$)*1024
  ELSE
    RETURN VAL(a$)
  ENDIF
ENDFUNCTION
\end{verbatim}
\end{mdframed}
or 
\begin{mdframed}[hidealllines=true,backgroundcolor=blue!20]
\begin{verbatim}
DEFFN ob_x(adr%,idx%)=DPEEK(adr%+24*idx%+16)
DEFFN ob_y(adr%,idx%)=DPEEK(adr%+24*idx%+18)
DEFFN ob_w(adr%,idx%)=DPEEK(adr%+24*idx%+20)
DEFFN ob_h(adr%,idx%)=DPEEK(adr%+24*idx%+22)
\end{verbatim}
\end{mdframed}

\paragraph{obsolete, because TOS-Specific, and no similar function on other OSes exist:}


\begin{tabbing}
XXXXXXXXXXXXXXXXX\=XXXXXXXXXXXXXXXXXX\=\kill\\
\verb|ADDRIN, ADDROUT|\> address of the AES Address Input/Output blocks\\
\verb|APPL_EXIT()|\>	Declare program has finished\\
\verb|APPL_INIT()|\>	Announce the program as an application.\\
\verb|APPL_TPLAY()|\>	Plays back a record of user activities\\
\verb|APPL_TRECORD()|\>	makes a record of user activities\\
\verb|BIOS()|\>		call BIOS routines\\
\verb|CONTRL|\>		Address of the VDI control table.\\
\verb|FGETDTA()|\>	Returns the DTA (Disk Transfer Address).\\
\verb|FSETDTA|\> 	Sets the address of the DTA\\
\verb|GB, GCONTRL |\>    Address of the AES Parameter/control Block\\
\verb|GDOS?|\>		Returns  TRUE  (-1)  if GDOS is  resident\\
\verb|GEMDOS()|\>	call the GEMDOS routines.\\
\verb|GEMSYS|\>		call the AES routine\\
\verb|GINTIN, GINTOUT|\> address of the AES Integer input/output block.\\
\verb|HIMEM |\>         address of the area of memory which is not allocated\\
\verb|INTIN, INTOUT|\>   address of the VDI integer Input/output block.\\
\verb|L~A |\>           	Returns base address of the LINE-A variables.\\
\verb|MENU_REGISTER()|\> Give a desk accessory a name\\
\verb|MONITOR|\>        	Calls a monitor resident in memory.\\
\verb|SHEL_GET, SHEL_PUT|\>  	obsolete\\
\verb|SHEL_WRITE|\>	obsolete\\
\verb|VDIBASE, VDISYS|\> VDI  functions\\
\verb|VQT_EXTENT|\>     	coordinates of a rectangle  which surround the text\\
\verb|VQT_NAME()|\>     	VDI  function\\
\verb|VSETCOLOR |\>     	TOS specific\\
\verb|VST_LOAD_FONTS(), VST_UNLOAD_FONTS()|\\
\verb|V_CLRWK(), V_CLSVWK(), V_CLSWK(), V_OPNVWK()|	\\
\verb|V_OPNWK(), V_UPDWK()|\> VDI-GDOS functions\\
\verb|V~H|\>	       	Returns the internal VDI handle\\
\verb|W_HAND(#n) |\>    	Returns the GEM handle of the window\\
\verb|W_INDEX()  |\>    	Returns the window number of the specified GEM handle.\\
\verb|WORK_OUT() |\>    	Determines the values from OPEN\_WORKSTATION.\\
\verb|XBIOS()|\>	       	call XBIOS system routines.\\

\end{tabbing}

\paragraph{Obsolete, because ATARI-ST-Hardware-Specific, and no similar function
exists on UNIX or SDL platforms:}

\begin{tabbing}
XXXXXXXXXXXXXXXXXXXX\=XXXXXXXXXXXXXXXXXX\=\kill\\
\verb|CHDRIVE|\>	Sets the default disk drive --> \verb|CHDIR|\\
\verb|DMACONTROL, DMASOUND|\>     Controls the DMA sound (see \verb|PLAYSOUND|)\\
\verb|INPMID$|\>        	read data from the MIDI port\\
\verb|LPENX, LPENY |\>   Returns the coordinates of a light pen.\\
\verb|PADT(), PADX(), PADY()|\> Reads the paddle on the STE\\
\verb|SDPOKE, SLPOKE, SPOKE|\>  Supervisor mode memory access\\

\end{tabbing}

\paragraph{Not supported because of other reasons:}

\begin{tabbing}
XXXXXXXXXXXXXXXXX\=XXXXXXXXXXXXXXXXXX\=\kill\\
\verb|APPL_READ()|\>	read from the event buffer.\\
\verb|APPL_WRITE()|\>	write to the event buffer.\\
\verb|BASEPAGE|\>	address of the basepage\\
\verb|BITBLT|\>		Raster copying command\\
\verb|CFLOAT()|\>	Changes integer into a floating point number.\\
\verb|DEFBIT, DEFBYT, DEFWRD, DEFFLT, DEFSTR |\>  \>      sets the variable  type\\
\verb|DFREE()|\>		free space on a disc\\
\verb|GETSIZE()|\>	return the number of Bytes required by a screen area\\
\verb|HARDCOPY|\>       	Prints the screen          --> save screen\\
\verb|INPAUX$|\>        	read data from the serial port\\
\verb|KEYDEF |\>        	Assign a string to a Function Key.\\
\verb|LLIST |\>         	Prints out the listing of the current program.\\
\verb|LPOS()|\>         	column in which the printer head is located\\
\verb|LPRINT|\>         	prints data on the printer.\\
\verb|PSAVE |\>         	save with protection\\
\verb|RCALL |\>    	Calls an assembler routine\\
\verb|SCRP_READ()|\>   communication between GEM programs.\\
\verb|SCRP_WRITE()|\>   "\\
\verb|SETCOLOR i,r,g,b|\>  set rgb value of color cell (-->GET\_COLOR())\\
\verb|SETTIME|\>		Sets the time and the date.\\
\verb|WINDTAB|\>	       	Gives the address of the Window Parameter Table.\\
\verb|WIND_CALC(), WIND_CLOSE(),| \\
\verb|WIND_CREATE(), WIND_DELETE(), WIND_FIND(),| \\
\verb|WIND_GET(), WIND_OPEN(), | \\
\verb|WIND_SET(), WIND_UPDATE()|\> \> GEM-Window-Function\\
\end{tabbing}

\subsubsection*{These GFA-Basic commands may be supported in a later version of 
X11-Basic:}

\begin{tabbing}
XXXXXXXXXXXXXXXXX\=XXXXXXXXXXXXXXXXXX\=\kill\\
\verb|APPL_FIND(fname$)|\> Returns the ID of the sought after application.\\
\verb|BYTE{x}|\>		read the contents of the address x\\
\verb|C:|\>	      Calls a C or assembler program with parameters as in C\\
\verb|CARD{x}|\>	      Reads/writes a 2-byte unsigned integer\\
\verb|CHAR{x}|\>	      Reads a string of bytes until a null byte is  encountered\\
\verb|DEFLIST x|\>       Defines the program listing format.\\
\verb|DEFNUM n|\>        Affects  output  of numbers by the PRINT  command\\
\verb|DELETE x(i)|\>     Removes  the i-th element of array x.\\
\verb|DO UNTIL |\>      extension\\
\verb|DO WHILE |\>      extension\\
\verb|DOUBLE{x}|\>       reads/writes  an  8-byte floating point  variable\\
\verb|EVNT_BUTTON()|\>   Waits for one or more mouse clicks\\
\verb|EVNT_DCLICK()|\>   Sets the speed for double-clicks of a mouse button.\\
\verb|EVNT_KEYBD()|\>    Waits for  a key to be pressed and  returns  a  word\\
\verb|EVNT_MESAG()|\>    Waits for the arrival of a message in the event buffer.\\
\verb|EVNT_MOUSE()|\>    Waits for  the  mouse pointer to  be  located  inside \\
\verb|EVNT_MULTI()|\>    Waits for the occurrence of selected  events.\\
\verb|EVNT_TIMER() |\>   waits for a period  of  time\\
\verb|FATAL|\>	      Returns the value 0 or -1 according to the type of error\\
\verb|FIELD|\>	      Divides records into fields.\\
\verb|FORM INPUT|\>      Enables the insertion of a character string\\
\verb|FORM INPUT AS|\>   the current value of a\$ is displayed, and can be edited.\\
\verb|FORM_BUTTON()|\>   Make inputs in a form possible using the mouse.\\
\verb|FORM_ERROR()|\>    Displays the ALERT associated with the error numbered\\
\verb|FORM_KEYBD()|\>    Allows a form to be edited via the keyboard.\\
\verb|FSEL_INPUT()|\>    Calls the AES fileselect library\\
\verb|GET #|\>	      Reads a record from a random access file.\\
\verb|GRAF_DRAGBOX()|\>  a rectangle to be moved about the screen\\
\verb|GRAF_GROWBOX()|\>  Draws an expanding rectangle.\\
\verb|GRAF_HANDLE() |\>  supplies the size of a character from the system set\\
\verb|GRAF_MKSTATE()|\>  supplies the current mouse  coordinates\\
\verb|GRAF_MOUSE)   |\>  allows  the  mouse  shape  to  be  changed.\\
\verb|GRAF_MOVEBOX()|\>  a moving rectangle with constant size\\
\verb|GRAF_RUBBERBOX()|\> draws an outline of a rectangle\\
\verb|GRAF_SHRINKBOX()|\> Draws an shrinking rectangle.\\
\verb|GRAF_SLIDEBOX()|\>  moves one rectangular object within another\\
\verb|GRAF_WATCHBOX()|\>  monitors  an object tree while a mouse button  is  pressed\\
\verb|HTAB|\>	       Positions  the cursor to the  specified  column. \\  
\verb|OUT?()|\>  output to device\\
\verb|INSERT|\>	       Inserts  an  element  into  an  array.\\
\verb|INT{x}|\>	       Reads/writes a 2 byte signed integer from/to address x.\\
\verb|KEYGET n |\>        similar to INKEY\$  but wait\\
\verb|KEYLOOK n |\>       similar to INKEY\$  but put back chars\\
\verb|KEYTEST n |\>       similar to INKEY\$\\
\verb|KEYPAD n  |\>       Sets  the  usage of the  numerical  keypad.\\
\verb|KEYPRESS n|\>       This  simulates the pressing of a key.\\
\verb|LONG{x}|\>	       Reads/writes a 4 byte integer from/to address x. Abbrev {}\\
\verb|LOOP UNTIL condition|\>     extension\\
\verb|LOOP WHILE condition|\>     extension\\
\verb|LSET var=string|\>  Puts the 'string' in the string variable 'var' left justified\\
\verb|MAT ADD a(),b()|\>\\
\verb|MAT ADD a(),x|\>\\
\verb|MAT CPY a([i,j])=b([k,l])[,h,w]|\>\\
\verb|MAT INPUT #i,a()|\>\\
\verb|MAT MUL|\>\\
\verb|MAT MUL a(),x|\>\\
\verb|MAT MUL x=a()*b()|\>\\
\verb|MAT MUL x=a()*b()*c()|\>\\
\verb|MAT NORM a(),{0/1}|\>\\
\verb|MAT PRINT [#i]a[,g,n]|\>\\
\verb|MAT QDET x=a([i,j])[,n]|\>\\
\verb|MAT RANG x=a([i,j])[,n]|\>\\
\verb|MAT READ a()|\>\\
\verb|MAT SET a()=x|\>\\
\verb|MAT SUB a(),b()|\>\\
\verb|MAT SUB a(),x|\>\\
\verb|MAT XCPY a([i,j])=b([k,l])[,h,w]|\>\\
\verb|MAT BASE {0/1}|\>\\
\verb|MENU(x)|\>	      Returns the information about an event in the variable \\
\verb|MENU OFF|\>        Returns a menu title to 'normal' display.\\
\verb|MENU_BAR()|\>      Displays/erases a menu bar (from a resource file)\\
\verb|MENU_ICHECK()|\>   Deletes/displays a tick against a menu item.\\
\verb|MENU_IENABLE()|\>  Enables/disables a menu entry.\\
\verb|MENU_TEXT()|\>     Changes the text of a menu item.\\
\verb|MENU_TNORMAL()|\>  Switches the menu title to normal/inverse video.\\
\verb|MID$(a$,x[,y])=|\> (as a command/lvalue)\\
\verb|MODE|\>	      representation of decimal point, date and files\\
\verb|OBJC_CHANGE()|\>        Changes the status of an object.\\
\verb|OBJC_EDIT()|\>	      Allows input and editing\\
\verb|OBJC_ORDER()|\>	      re-positions an object within a tree.\\
\verb|OB_ADR()|\>	      Gets the address of an individual object.\\
\verb|OB_FLAGS()|\>	      Gets the status of the flags for an object.\\
\verb|OB_H()|\> 	      Returns the height of an object\\
\verb|OB_HEAD()|\>	      Points to the object's first child\\
\verb|OB_NEXT()|\>	      Points to the following object on the same level\\
\verb|OB_SPEC()|\>	      Returns the address of the data structure\\
\verb|OB_STATE()|\>	      returns the status of an object\\
\verb|OB_TAIL()|\>	      Points to the objects last child\\
\verb|OB_TYPE()|\>	      Returns the type of object specified.\\
\verb|OB_W()|\> 	      Returns the width of an object\\
\verb|OB_X(), OB_Y()|\>       relative  coordinates  of  the   object\\
\verb|ON BREAK|\>	      influence behavior of CTRL-C\\
\verb|OPTION BASE|\>	      determine whether an  array  is  to contain a zero element\\
\verb|RCALL|\>        Calls  an assembler routine\\
\verb|RC_INTERSECT()|\>  Detects whether two rectangles overlap.\\
\verb|RECALL|\>       Inputs n lines from a text file\\
\verb|RECORD |\>	      Sets the number of the next record (GET\#, PUT\#)\\
\verb|RSET a$=b$|\>   Moves a string expression, right justified to a string.\\
\verb|RSRC_OBFIX()|\> converts the coordinates of an object\\
\verb|RSRC_SADDR()|\> sets the address of an object.\\
\verb|RUN <filename>|\>   see RUN\\
\verb|SETDRAW|\>      see DRAW\\
\verb|SINGLE{x}|\>    Reads/writes  a  4  byte floating point\\
\verb|SPRITE|\> 	      Puts a sprite\\
\verb|STORE|\>  	      Fast save of a string array as a text file.\\
\verb|TRON# |\> 	      Tron to file\\
\verb|TRON proc |\>    procedure is called  before  the execution of each command\\
\verb|VTAB|\>         positions the cursor to the specified line number\\
\verb|WRITE|\>  	      Stores data in a sequential file\\
\verb|_DATA|\>	       	Specifies the position of the DATA pointer.\\
\verb|STICK|\>		control the joystick (via SDL only)\\
\end{tabbing}

\subsubsection*{Following commands have a different meaning and/or syntax in X11-Basic:}


\begin{center}\begin{tabular}{|ll|}
\hline
{\bf GFA-BASIC }        & {\bf X11-Basic}\\
\hline
\verb|SYSTEM|           & \verb|QUIT|\\
\verb|LINE INPUT|       & \verb|LINEINPUT|\\
\verb|SOUND|		& \verb|SOUND|\\
\verb|WAVE|		& \verb|WAVE|\\
\verb|VSYNC|            &  -\\
\verb|ON MENU|		& \verb|MENU|\\
\verb|ON MENU GOSUB ...|& \verb|MENUDEF|\\
\verb|MENU a$()|        & \verb|MENUDEF|\\
\verb|MENU OFF|         &  -\\
\verb|MENU KILL|        & \verb|MENUKILL|\\
\verb|MENU()|           &  -\\
\verb|MONITOR|          & \verb|SYSTEM|\\
\verb|EXEC|             & \verb|EXEC|\\
\verb|RENAME AS|        & \verb|RENAME|\\
\hline
\end{tabular}\end{center}

\subsubsection*{Compiler specifics}

\begin{itemize}
\item \verb|PRINT| statements will not compile correctly sometimes. Avoid to use functions and variables in print statements, which are not used anywhere else. 
\item \verb|ON ERROR GOSUB| will not work correctly in compiled programs.
\item \verb|ON ERROR GOTO| will not work correctly in compiled programs.
\item \verb|ON BREAK GOSUB| will not work correctly in compiled programs.
\item \verb|ON BREAK GOTO| will not work correctly in compiled programs.
\end{itemize}

%\subsubsection*{Differences between UNIX, WINDOWS, TomTom and Android versions}
%
% uff, hier koennten wir nochwas schreiben, aber ich bin zu faul...
%

\section{Ideas for future releases of X11-Basic}

These are some ideas for new commands, which are not GFA-commands and 
which might be implemented in X11-Basic in future:

\begin{verbatim}

 SPRINT var$;[USING...;]...  similar to sprintf() in C 
 MAT_PRINT or PRINT a()

========================
implementation of mmap() in X11-Basic:

open "I",#1,"myfile"
adr%=map("I|O|U",#1,len,offset)


MSYNC adr%,len

UNMAP adr%,len
CLOSE #1

       offset should  be a multiple of the page size 
       as returned by getpagesize().

"I"   --> PROT_READ MAP_PRIVATE
"O"   --> PROT_WRITE MAP_SHARED
"U"   --> PROT_READ PROT_WRITE MAP_SHARED

"*L"  --> MAP_LOCKED

=============================================

modifiable lvalues:
MID$()=
CHAR{}=
PRG$()=
new command (for threads):

EXSUB (instead of gosub) procedure

alternative:
FIRE procedure()
or
KICK procedure()

SPAWN  ....

it must be guaranteed that the program flow control and 
the access to variables etc, is thread-safe. 
This might be difficult....


=============================================
USB support:  (not completely done, I need someone 
                who uses this for testing...)

OPEN "UU",#2,devicename%,vid,pid,class,endpoint
SEND_CONTROL #2,t$
SEND #2,t$
RECEIVE #2,t$
RECEIVE_BULK #2,t$
EOF(#2)
INP?(#2)
CLOSE #2


=============================================
SQL support ???

use the sqlite excecutable and SYSTEM$()

=============================================
arbitrary precision (floatingpoint) numbers
** integers and floatingpoint/complex numbers
a##=1.84902948755588584888888888888888834

a|=


we need new parsers, type guessing routine,
all operations need to work, complex functions..... Casts...

=============================================
Support for the gcrypt library.
-------encryption---------------

LIBGCRYPT:
    hash$=HASH$(data$[,typ%])   (done)
    sdata$=SIGN$(data$,privkey$)   (done)
    verify%=VERIFY$(sdata$,pubkey$)   (done)
    cdata$=ENCRYPT$(data$,key$[,typ%])   (done)
    data$=DECRYPT$(cdata$,key$[,typ%])   (done)

    err=KEYGEN(typ%,pubkey$,privkey$)
\end{verbatim}
